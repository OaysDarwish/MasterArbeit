\section{AgileVSWasserfallmodell}

Agile vs. Wassfallmodell:

Wasserfallmodell:
Lange Zeit galt das Wasserfallmodell durch lineare, also aufeinanderfolgende Projektphasen als Idealtyp im Bereich der Anwendungsentwicklung. Nur zögerlich wurde der Ansatz gegen agile Methoden ausgetauscht, allerdings sind die Anforderungen für dieses Modell auch sehr hoch. Ursprünglich aus dem Produktionsprozess stammend, hat sich das Wasserfallmodell sehr schnell in der Softwareentwicklung etablieren können. Es ist ein lineares Modell für Entwicklungsprozesse, welches eine klare Kontrolle der jeweils erreichten Aktivitäten und Meilensteine erlaubt und somit die Kontrolle über das Projekt äußerst streng regelt. Die meisten Wasserfallmodelle arbeiten mit fünf oder sechs Phasen. Diese werden bereits zu Beginn klar definiert, allerdings ist die erste Phase zugleich auch die Planungsphase und Grundlage für den weiteren Erfolg. unterteilt sich dies in der Regel in folgende Phasen:

Planung

Definition

Entwurf

Implementierung

Tests

Auslieferung

Agile Produktentwicklung:
Agile Softwareentwicklung soll dafür sorgen, dass Entwicklungsprojekte einfach, unbürokratisch und iterativ ablaufen. Agile Methoden und Prozesse wie Scrum können somit Zeit und Kosten sparen.
Den Begriff „Agile Softwareentwicklung“ gibt es erst seit Anfang der 2000er Jahre, jedoch sind erste Aspekte dieser Methode schon in den frühen Neunzigern zu verzeichnen. Das vom Informatiker Kent Beck im Jahr 1999 veröffentlichte Buch „Extreme Programmierung“ bildete schließlich die theoretische Grundlage für die Beschreibung agiler Prozesse. Kent Beck und sein Forschungsteam waren es auch, die 2001 das sogenannte „Agile Manifesto“ veröffentlichten, in dem die Grundprinzipien der agilen Softwareentwicklung festgehalten sind.
Das Agile Manifesto enthält insgesamt zwölf Prinzipien, die es bei der Programmierung und Softwareentwicklung zu beachten gilt. Die Kunden sollen zum einen durch die schnelle Entwicklung einer funktionierenden Software zufrieden gestellt werden. Zum anderen sollen sie durch kontinuierliche Veränderungen während des Entwicklungsprozesses einen Wettbewerbsvorteil erhalten. Ein weiteres Prinzip lautet, dass Fachleute und Entwickler während der einzelnen Projektphasen intensiv zusammenarbeiten sollten. Außerdem soll für die am Softwareprojekt Beteiligten ein motivierendes Arbeitsumfeld geschaffen werden. Die Kommunikation unter den Projektbeteiligten soll möglichst Face-to-Face erfolgen. Um diese Prinzipien umzusetzen, setzt man bei einem Softwareprojekt verschiedene agile Prozesse ein. Zu den beliebtesten agilen Prozessen zählt Scrum. Bei Scrum handelt es sich um eine Form des Projektmanagements, die auf Projektmanager verzichtet. Auch die Projektmanagement-Methoden Kanban wird häufig eingesetzt. Mit Hilfe des Kanban-Prinzips lässt sich das Aufgabenmanagement agiler gestalten.
Den richtigen Ansatz finden:
Beim Evaluieren von agilem Projektmanagement im Vergleich zum Wasserfallmodell hat jeder Ansatz seine Vor- und Nachteile. Mit dem agilen Prozess kann man neue, faszinierende Funktionen wesentlich schneller entwickeln, allerdings wird jede Änderung auch einige Zeit für die Fehlerbehebung in Anspruch nehmen. Mit einem strenger durchorganisierten Wasserfallprozess werden die neuen Versionen besser vollendet. Was es der Führungskraft (rein theoretisch) erleichtert, das Budget, den Zeitaufwand und die Arbeit, die für die Fertigstellung des Projektes notwendig ist, zu kalkulieren.
Aus Kundensicht führt die kontinuierliche agile Entwicklung dazu, dass die Produkte, die sie bereits nutzen, schrittweise verbessert werden und verstärkt die Annahme von Modellen wie monatlichen Abonnements. Wenn nach dem Wasserfallmodell entwickelt wird, führt zwar dazu, dass die Erwartungen der Kunden an neue Produkte und Innvationen etwas limitiert sind, aber es beflügelt die Vorfreude auf das neue Produkt. Außerdem können sich Teams, die nach der Wasserfallmethode arbeiten, mehr darauf konzentrieren, dass das Endprodukt auch wirklich den Ansprüchen der Kunden entspricht und ihnen eine positive Erfahrung bietet.
Beide Methoden gleichzeitig anwenden:
Wrike ist eine leistungsstarke Projektmanagement-Lösung, die Führungskräften ermöglicht, die Methode zu nutzen, die sich am besten eignet, um ihre Ziele zu erreichen. Sie können einen Wasserfall-Arbeitsablauf mit einem Gantt-Diagramm einrichten oder mit Dashboards und benutzerdefinierten Arbeitsabläufen einen agilen Arbeitsablauf erstellen. Egal mit welcher Methode Sie arbeiten, mit Wrike können Sie die Prozesse genau überwachen, durch eine Projekt-Zeitleiste, kontinuierliche Kommunikation mit Ihrem Team oder Stakeholdern und die Möglichkeit, auf einfache und bequeme Art und Weise Informationen und Dokumente zu teilen.

