\chapter{Einleitung}
%
%
%
%
%
%
Jedes Jahr wird das Leben von etwa 1,3 Millionen Menschen durch einen Verkehrsunfall beendet. Zwischen 20 und 50 Millionen weitere Menschen erleiden nicht-tödliche Verletzungen, wobei viele infolge ihrer Verletzung eine Behinderung erleiden. Verkehrsunfälle verursachen erhebliche wirtschaftliche Verluste für Einzelpersonen, ihre Familien und Nationen insgesamt. Diese Verluste ergeben sich aus den Behandlungskosten sowie aus Produktivitätsverlusten für diejenigen, die durch ihre Verletzungen getötet oder behindert wurden, und für Familienmitglieder, die sich von der Arbeit oder Schule freinehmen müssen, um sich um die Verletzten zu kümmern. Straßenverkehrsunfälle kosten die meisten Länder 3\% ihres Bruttoinlandsprodukts. Verletzungen im Straßenverkehr sind die häufigste Todesursache für Kinder und junge Erwachsene im Alter von 5 bis 29 Jahren.\citep{healthorganization2022}

Motorradfahren findet in Deutschland immer mehr Zuwachs.  Rund 3,8 Millionen zweirädrige Kraftfahrzeuge waren am 1. Januar 2012 in Deutschland zugelassen. Dies entspricht 7,26 \% aller zugelassener Kraftfahrzeuge in Deutschland \citep{Haedrich2012}.

Motorradfahrer stellen im Straßenverkehr eine besonders gefährdete Gruppe an Verkehrsteilnehmern dar. Studien aus dem Jahr 2014 zeigten, dass die Wahrscheinlichkeit in den USA ein Motorradfahrer bei einem Unfall zu sterben, 27-mal höher ist als die eines Autofahrers, und dass Verletzungen sechsmal so wahrscheinlich sind \citep{NHTSA}.

Verzögerungen bei der Erkennung und Versorgung der an einem Verkehrsunfall beteiligten Personen erhöhen die Schwere der Verletzungen. Die Versorgung von Verletzungen nach einem Unfall ist äußerst zeitkritisch. Verzögerungen von Minuten können über Leben und Tod entscheiden. Die Verbesserung der Versorgung nach einem Unfall erfordert die Sicherstellung des Zugangs zu rechtzeitiger präklinischer Versorgung und die Verbesserung der Qualität sowohl der präklinischen als auch der stationären Versorgung.\citep{healthorganization2022}

Die Reaktionszeit der Rettung spielt eine besonders große Rolle dabei, Leben zu retten.
-> Unfallerkennungsalgorithmus; \\
-> Automatische Unfallerkennung \\
-> kürzere Reaktionszeit; Genauere Informationen mitgeteilt; .... usw.\\
-> kurze Versorgungszeit -> Leben retten!!\\

.\\

-	Geschwindigkeitsüberschreitung bei Motorrädern viel häufiger als bei Autos.\\

.\\

Diese Arbeit beschäftigt sich mit der Weiterentwicklung eines Unfallerkennungsalgorithmus mittels Smartphone. \\
- Pocket-Mode\\
- Edgecases\\
- Mögliche Maßnahmen\\
- 











