\chapter{Ergebnisse}


- Auswertung (Qualitative Auswertung)\\
- Lauferkennung \\
- Rechenzeit: kein wesentlicher Unterschied (55 Sec und 57 Sec)

\section{Verifikationsversuche}

- Alle geplanten Szenarien wurden mind. einmal getestet.\\

- Die erste sowie letzte Szenarien wurden jeweils zweimal getestet.\\


\section{Lauferkennung}

\begin{figure}[H]
	\centering
	\includegraphics[width=\linewidth]{Bilder/Speed_Groundtruth_MotionClass_GoogleMD_Compare.png}
	\caption{Ergebnis des Lauferkkennungsmodells}
	\label{fig:Speed_Groundtruth_MotionClass_GoogleMD_Compare}
\end{figure}



\section{Verschiedene Fahrerpositionierung}

\begin{figure}[H]
	\centering
	\includegraphics[width=\linewidth]{Bilder/Speed_Groundtruth_WalkStand_Compare.png}
	\caption{Vergleich der Fahrt in vertikaler und normaler Position}
	\label{fig:Speed_Groundtruth_WalkStand_Compare}
\end{figure}

\section{Anhalten}

\begin{figure}[H]
	\centering
	\includegraphics[width=\linewidth]{Bilder/Speed_AngleChangeCompare.png}
	\caption{Winkeländerung des Smartphones durch das Anhalten}
	\label{fig:Speed_AngleChangeCompare}
\end{figure}
Winkeländerung nicht über \ang{45} beträgt $->$ kein Fehlalarm.\\
\\
Beurteilung:

Zu diesem Zweck wurden mehreren Tests durchgeführt, während Diesen keine falsche Alarmauslösungen aktiviert wurden. Grund ist, dass die Winkeländerung nicht \ang{90} beträgt sondern nur ca. \ang{20}-\ang{30} (\autoref{fig:MotorbikeDrivingStanding}). Die Person hat sein Bein beim Sitzen nicht genau horizontal sondern leicht nach Unten geneigt. Und wenn der Fahrer sein Fuß runter setzt, ist diese auch nicht genau vertikal sondern bisschen gebogen mit einem Winkel von ca. \ang{10}-\ang{20} zum Vertikallinie. D.h. die Winkeländerung ist nicht über \ang{45} und sollte zu keinen Alarmauslösungen führen..

\begin{figure}[H]
	\centering
	\begin{subfigure}{\textwidth}
		\centering
		\includegraphics[width=0.5\textwidth]{Bilder/MotorbikeDriving2.png}
		\caption{Beinposition während einer Fahrt}
		\label{fig:MotorbikeDriving}
	\end{subfigure}
	\hfill
	\begin{subfigure}{\textwidth}
		\centering
		\includegraphics[width=0.5\textwidth]{Bilder/MotorbikeStanding2.png}
		\caption{Beinposition Beim Stehen}
		\label{fig:MotorbikeStanding2}
	\end{subfigure}
	\caption{Beinpositionen während einer Fahrt und gestreckt}
	\label{fig:MotorbikeDrivingStanding}
\end{figure}


\section{Vor- und Nachteile}
- Rechenzeit: kein wesentlicher Unterschied (55 Sec und 57 Sec)





 