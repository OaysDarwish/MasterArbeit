\chapter{Ergebnisse}

In diesem Kapitel wird die implementierte Lauferkennung ausgewertet und werden ebenfalls die Ergebnisse der Testversuchen vorgestellt und diskutiert.

Die Auswertung der Versuche beziehungsweise der Testergebnisse wird aus zeitlichen Gründen qualitativ sein und nicht quantitativ.


- Rechenzeit: kein wesentlicher Unterschied (55 Sec und 57 Sec)

\section{Verifizierungsversuche}

Die bereits im \autoref{ab:Versuchsplanung} aufgelistete Versuchsszenarien wurden alle mindestens einmal getestet. Einige Testszenarien sind mit zwei Testdurchläufe durchgeführt.
Nach der Durchführung der Verifizierungsversuche werden die Ergebnisse analysiert und diskutiert. 

Die Lauferkennung ist ein großer Teil dieser Arbeit und wird demnächst mit den tatsächlichen Daten sowie die von Google integrierte Aktivitätserkennung verglichen.
\section{Lauferkennung}
In der \autoref{fig:Speed_Groundtruth_MotionClass_GoogleMD_Compare} werden die verschiedenen Methoden der Aktivitätserkennung sowie die Geschwindigkeit in der obersten Grafik dargestellt.

Die zweite Grafik zeigt die tatsächliche Aktivitätsdaten (Ground Truth), die in vier Klassen (Unbekannt, Keine Bewegung, Laufen, Fahren) unterteilt sind (\autoref{fig:Lauferkennung_Klassentabelle}). Die Aktivitätsbereiche sind auch mit roten Linien optisch unterschieden, die auch bei anderen Grafikteilen sichtbar sind, damit diese einfach und schnell zugeordnet werden können. Bei der Sekunde $460$ s sowie $500$ s ist die Versuchsperson vom Motorrad ab- oder aufgestiegen, deswegen wurden diese zwei Stellen mit '"No Motion"' klassifiziert. Ab der Sekunde $556$ gibt es keine Videoaufnahme mehr, deswegen wurde als "'Undefined"' eingetragen.

Es ist zu bemerken, dass die Lauferkennung ein sehr große Übereinstimmung mit den Ground Truth hat. Laufen sowie Fahren werden gut erkannt. Das Auf- und Absteigen bei den Sekunden $465$ s $508$ s werden die "'No Motion"'-Klasse vom Lauferkennung-Modell vergeben.

In der dritten Grafik ist die Ausgabe der im Rahmen dieser Arbeit implementierten Lauferkennung abgebildet. 
Die letzte Grafik stellt die Ausgabe der im Smartphone integrierten Aktivitätserkennung dar, die für diese Auswertung zu der gleichen Klassifizierung umgerechnet wurde.

Diese drei Grafiken unterscheiden zwischen drei Klassen:

Die Kurve hat die gleiche Klassifizierung wie die vorherige Grafik, damit diese mit einander einfach verglichen werden können.


\begin{figure}[H]
	\centering
	\includegraphics[width=\linewidth]{Bilder/Speed_Groundtruth_MotionClass_GoogleMD_Compare.png}
	\caption{Ergebnis des Lauferkkennungsmodells}
	\label{fig:Speed_Groundtruth_MotionClass_GoogleMD_Compare}
\end{figure}

\begin{figure}[H]
	\centering
	\includegraphics[width=0.6\linewidth]{Bilder/Lauferkennung_Klassentabelle.png}
	\caption{Beschreibung der Klassen in der Grafik \autoref{fig:Speed_Groundtruth_MotionClass_GoogleMD_Compare}}
	\label{fig:Lauferkennung_Klassentabelle}
\end{figure}

Aus der Grafiken ist zu bemerken, dass die Klassifizierung durch das implementierte Modell sehr gute Ergebnisse liefert, die mit der Wahrheit mehr als $85\%$ übereinstimmt. Im Vergleich zu der Google-Aktivitätserkennung hat das Modell eine wesentlich bessere Aussage getroffen.



\section{Verschiedene Fahrerpositionierung}
Die \autoref{fig:Speed_Groundtruth_MotionClass_GoogleMD_Compare} zeigt vier verschiedene Grafiken mit der gleichen x-Achse (Zeit [Sec]).
In der ersten Grafik ist die Geschwindigkeit (blau) sowie die Alarmauslösung (lila) über die Zeit abgebildet.
Die zweite Grafik stellt die tatsächliche Daten der Fahrerposition (Sitzen, Stehen oder unbekannt) über die Zeit dar.
In der dritten Grafik ist der tatsächliche Verlauf Fahreraktivität (Fahren, Laufen oder unbekannt) dargestellt.
Die Kurve aus der letzten Grafik zeigt den Verlauf des Gerätswinkels im Vergleich zur ursprünglichen Platzierung während der Kalibrierung. 

\begin{figure}[H]
	\centering
	\includegraphics[width=\linewidth]{Bilder/Speed_Groundtruth_WalkStand_Compare.png}
	\caption{Vergleich der Fahrt in vertikaler und normaler Position} %TODO: ausführilicher Erklärung im Unterschrifft
	\label{fig:Speed_Groundtruth_WalkStand_Compare}
\end{figure}


Aus der \autoref{fig:Speed_Groundtruth_MotionClass_GoogleMD_Compare} ist Folgendes zu verstehen:

\begin{itemize}
	\item In der Zeitbereich zwischen $388 - 458$ s findet eine Fahrt statt, während dieser die Fahrerposition sich verändert hat. Der Fahrer ist für ca. $10$ s gestanden und wieder gesessen.
	\item In der Zeitbereich zwischen $465 - 505$ s ist die Versuchsperson gelaufen, deswegen lautet die Klassifizierung in der zweiten Grafik '"Unbekannt"', da die gedachte Klassen so ein Fall nicht abdecken.
	\item  
\end{itemize}

\section{Anhalten}

\begin{figure}[H]
	\centering
	\includegraphics[width=\linewidth]{Bilder/Speed_AngleChangeCompare.png}
	\caption{Winkeländerung des Smartphones durch das Anhalten}
	\label{fig:Speed_AngleChangeCompare}
\end{figure}
Winkeländerung nicht über \ang{45} beträgt $->$ kein Fehlalarm.\\
\\
Beurteilung:

Zu diesem Zweck wurden mehreren Tests durchgeführt, während Diesen keine falsche Alarmauslösungen aktiviert wurden. Grund ist, dass die Winkeländerung nicht \ang{90} beträgt sondern nur ca. \ang{20}-\ang{30} (\autoref{fig:MotorbikeDrivingStanding}). Die Person hat sein Bein beim Sitzen nicht genau horizontal sondern leicht nach Unten geneigt. Und wenn der Fahrer sein Fuß runter setzt, ist diese auch nicht genau vertikal sondern bisschen gebogen mit einem Winkel von ca. \ang{10}-\ang{20} zum Vertikallinie. D.h. die Winkeländerung ist nicht über \ang{45} und sollte zu keinen Alarmauslösungen führen..

\begin{figure}[H]
	\centering
	\begin{subfigure}{\textwidth}
		\centering
		\includegraphics[width=0.5\textwidth]{Bilder/MotorbikeDriving2.png}
		\caption{Beinposition während einer Fahrt}
		\label{fig:MotorbikeDriving}
	\end{subfigure}
	\hfill
	\begin{subfigure}{\textwidth}
		\centering
		\includegraphics[width=0.5\textwidth]{Bilder/MotorbikeStanding2.png}
		\caption{Beinposition Beim Stehen}
		\label{fig:MotorbikeStanding2}
	\end{subfigure}
	\caption{Beinpositionen während einer Fahrt und gestreckt}
	\label{fig:MotorbikeDrivingStanding}
\end{figure}


\section{Vor- und Nachteile}
- Rechenzeit: kein wesentlicher Unterschied (55 Sec und 57 Sec)





 