\chapter{Unfallerkennung im Pocket-Mode}

- Unterschied zum aktuellen Ziel:\\
- Smartphone wird momentan am Lenker befestigt\\

Konkreter Unterschied zu meiner MA

\section{Kritische Unfallszenarien}

Liste der Edge- und usecases mit einer Erklärung, warum diese kritisch sind und einen Vorschlag, was man dagegen tun kann.



\section{Lauferkennung} \label{sec:Lauferkennung}

%\begin{table}\caption{Statistische Zahlen über Unfälle in Deutschland \cite{Verkehrsunfaelle_Fahrrad2017}} 
%	\centering
%	\begin{tabular}{|p{3.2cm}|>{\centering\arraybackslash}p{3.3cm}|>{\centering\arraybackslash}p{3.3cm}|>{\centering\arraybackslash}p{3.3cm}|}
	%		\hline
	%		\textbf{Jahr} & \textbf{Unfälle} & \textbf{Verunglückte} & \textbf{Getötete} \\
	%		\hline
	%		2000 & 382.949 & 511.577 & 7.503 \\
	%		\hline
	%		2005 & 336.619 & 438.804 & 5.361 \\
	%		\hline
	%		2010 & 288.297 & 374.818 & 3.648 \\
	%		\hline
	%		2014 & 302.435 & 392.912 & 3.377 \\
	%		\hline
	%		2015 & 305.659 & 396.891 & 3.459 \\
	%		\hline
	%		2016 & 308.145 & 399.872 & 3.206 \\
	%		\hline
	%		2017 & 302.656 & 393.492 & 3.180 \\
	%		\hline
	%	\end{tabular}
%	\label{tab:UnfallImJahren}
%\end{table}
%
\subsection{Peaks aufzählen}


\subsection{Frequenzbasierend}
Auf die FFT beziehen

\subsubsection{Werte Sinnvoll auswählen}
- gültige Intensität\\
- Frequenzbereiche\\
- Gültige Geschwindigkeit\\
- Ausschlusskriterien\\





\section{Verifikation des Algorithmus'}
% Aufgrund der Verifizierung des Algorithmus werden im Folgenden statistische Zahlen über Fahrradunfälle weltweit sowie in Deutschland dargestellt. Danach wird eine statistische Fallzahlabschätzung durchgeführt, um die notwendige Anzahl der Versuche zu ermitteln. Eine Versuchsplannung wird anhand der Ergebnisse der Versuche durchgeführt, die bereits vor der eigentlichen Verifizierung durchgeführt wurden. 
%
%
%


\subsection{Groundtruth sammeln}

- Videos mit den Daten synchronisieren (Das Tool erläutern? oder vllt. nur erwähnen?)\\
- Groundtruth labels nachdenken und erläutern\\
- Videos labeln\\

















